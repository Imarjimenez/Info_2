%%%%%%%%%%%%%%%%%%%%%%%%%%%%%%%%%%%%%%%%%%%%%%%%%%%%%%%%%%%%%%%%%%%%%%%%%%%%%%%%
%2345678901234567890123456789012345678901234567890123456789012345678901234567890
%        1         2         3         4         5         6         7         8


\documentclass[letterpaper, 12 pt, conference]{ieeeconf}  % Comment this line out
                                                          % if you need a4paper
%\documentclass[a4paper, 10pt, conference]{ieeeconf}      % Use this line for a4
                                                          % paper




\IEEEoverridecommandlockouts                              % This command is only
                                                          % needed if you want to
                                                          % use the \thanks command
\overrideIEEEmargins
% See the \addtolength command later in the file to balance the column lengths
% on the last page of the document



% The following packages can be found on http:\\www.ctan.org
%\usepackage{graphics} % for pdf, bitmapped graphics files
%\usepackage{epsfig} % for postscript graphics files
%\usepackage{mathptmx} % assumes new font selection scheme installed
%\usepackage{times} % assumes new font selection scheme installed
%\usepackage{amsmath} % assumes amsmath package installed
%\usepackage{amssymb}  % assumes amsmath package installed

\title{\LARGE \bf
LA COMPUTACIÓN EMERGE DE CONCEPTOS RETUFABLES. 
}

%\author{ \parbox{3 in}{\centering Huibert Kwakernaak*
%         \thanks{*Use the $\backslash$thanks command to put information here}\\
%         Faculty of Electrical Engineering, Mathematics and Computer Science\\
%         University of Twente\\
%         7500 AE Enschede, The Netherlands\\
%         {\tt\small h.kwakernaak@autsubmit.com}}
%         \hspace*{ 0.5 in}
%         \parbox{3 in}{ \centering Pradeep Misra**
%         \thanks{**The footnote marks may be inserted manually}\\
%        Department of Electrical Engineering \\
%         Wright State University\\
%         Dayton, OH 45435, USA\\
%         {\tt\small pmisra@cs.wright.edu}}
%}

\author{Imar Jimenez${-1007424872 -INFORMATICA II - G2 }$% <-this % stops a space
%\thanks{I would like to Express my Thanks to }% <-this % stops a space

       }%
%\thanks{$^{}$
%        {\tt\small }}%
%}


\begin{document}



\maketitle
\thispagestyle{empty}
\pagestyle{empty}


%%%%%%%%%%%%%%%%%%%%%%%%%%%%%%%%%%%%%%%%%%%%%%%%%%%%%%%%%%%%%%%%%%%%%%%%%%%%%%%%
\begin{abstract}
    
"El razonamiento matemático puede considerarse más bien esquemáticamente como el ejercicio de una combinacion de dos instalaciones, que podemos llamar la intución y el ingenio " \textit{Alan Mathison Turing.}


\end{abstract}


%%%%%%%%%%%%%%%%%%%%%%%%%%%%%%%%%%%%%%%%%%%%%%%%%%%%%%%%%%%%%%%%%%%%%%%%%%%%%%%%
\section{INTRODUCCIÓN}

En el transcurso de la historia universal , el ser humano se ha destacado por ser un ente investigativo sobre muchos de los fenómenos  sociales,culturales, naturales y científicos en el mundo en el que está inmerso. Sin embargo, la indagación en dichos ámbitos provocó un estancamiento en el pensamiento crítico de la sociedad, ya que lo que se manifestaba estaba basado en afirmaciones cargadas de intuición. Pero empiezan a surgir múltiples dudas en una pequeña fracción de personas(matemáticos) sobre conceptos previamente planteados lo que deriva en el inicio de una de las premisas más importantes de la actualidad.

\section{Como todo empezó.}

\subsection{El infinito.}

La duda sobre la concepción del infinito emerge desde la civilización griega; Los antiguos griegos  plantean la idea del infinito desde su sentido común, lo que dio pie a que las contradicciones y las paradojas se convirtieran en protagonistas de este periodo. Esto causo que se crearan dos conceptos del infinito. El infinito potencial y el infinito actual, el infinito potencial consistia en una operacion reiterativa, inalcanzable e incomprensible, de ello nace el teorema de los polígonos inscritos y circunscritos. Por otro lado, el infinito actual a diferencia del potencial no se  refiere al infinito como un proceso, sino  al infinito como un todo o unidad, con ayuda de dicho modelo Leibniz(considerado fundador del cálculo junto a Newton) vio la derivada como una razón de diferencias infinitesimales.

Zenon englobó la idea del infinito potencial en su paradoja, en ella proclamó que el movimiento no existe al analizarlo con una serie infinitas de etapas.
A finales del siglo XIX, Georg Cantor desarrolló una teoría basada en el infinito actual, donde declara que existen conjuntos que poseen distintos números cardinales(número de elementos de un conjunto), es decir, existen infinitos más grandes que otros.



\section{Crisis de los fundamentos. }

A principios del siglo XX, surge un gran cambio en el ámbito matemático debido al flujo de diferentes pensamientos y demostraciones, lo que conlleva a revisar el sistema de intuiciones considerados como verdad. 

\subsection{David Hilbert.} 

Fue un matematico alemán  reconocido como el creador de la metamatemática.
En 1901 en los "Fundamentos de la Geometría", Hilbert formaliza la teoría axiomática. Que consiste en que los sistemas axiomáticos bien fundamentados tenían tres caracteristicas, consistencia, completitud y finitud.

\subsection{Kurt Gödel}
Nació en 1906 en Princeton,Estados Unidos. Es un lógico y matemático estadounidense. En 1931, cambia la percepción de nuestro sistema lógico "En el método que él propuso se establecen unas nociones básicas, se fijan unos axiomas y usando las reglas de la lógica, se van demostrando todas las verdades de las matemáticas". Pues bien, lo que vino a demostrar Gödel en su célebre teorema es que tal pretensión es un imposible: no pueden demostrarse todas las verdades de las matemáticas."\newline


   \begin{figure}[]
      \centering
      \framebox{\parbox{3in}{"No hay ningún método de prueba formal con el que poder demostrar todas las verdades de la matemática, y ni siquiera de la teoría elemental de los enteros positivos. Su prueba de este teorema, en sí misma estrictamente matemática, produjo un brusco giro en la filosofía de la matemática, pues habíamos supuesto que la verdad matemática consistía en la demostrabilidad". La ambición formalista de Hilbert era un imposible."\textit{ W. V. Quine.}

}}
      %\includegraphics[scale=1.0]{figurefile}
      \caption{}
      \label{}
   \end{figure}



\section{Nacimiento de la computación. }


\subsection{Enigma.}
Durante la segunda guerra mundial, los alemanes usaron una maquina de encriptacion llamada \textit{ Enigma.}.
El propósito era crear una máquina capaz de enviar una cantidad de datos sin ser descifrados por el territorio enemigo.
Se crearon cuatro modelos de dicha máquina. El más importante fue el Enigma-D, creado en 1926, fue tanto su éxito que se usó en la todas las organizaciones militares alemanes.
Este aparato es la combinación de sistemas mecánicos
y eléctricos. La parte mecánica está constituido por un teclado y un sistema de discos rotativos llamados rotores, colocados a lo largo de un huso y un mecanismo que hace girar uno o varios rotores cuando es pulsada una tecla. El continuo movimiento de los rotores provoca que la clave con la que se codifica el mensaje varíe cada vez que se oprima una tecla.En la variación se produce un circuito eléctrico diferente y el circuito encripta el mensaje.


\subsection{Alan Turing.}
 Nació en 1912 en Londres. En 1950 escribió el artículo \textit{"Computing machinery and intelligence"} donde relaciona el razonamiento humano con los procesos sistemáticos de una maquina. Lo mas sorprendente es que logró ensamblar sus ideas en la famosa \textit{"Maquina de Turing"}, lo más sorprendente fue la manera en como demostró se podía transformar simbolos y operaciones, en un sistema donde se plantea una serie de pasos que logran cumplir con un algoritmo propuesto.\newline
 
 
    \begin{figure}[]
      \centering
      \framebox{\parbox{3in}{"La maquina de Turing consisite en una cinta cuadriculada
infinita de una dimensión, un alfabeto finito de símbolos, un cabezal que puede
“leer” y “escribir” símbolos en cada uno
de esos cuadrados, y que puede moverse
de un cuadrado a su vecino a izquierda o
derecha, y un sistema de “estados”. La
operación de la máquina está codificada
esencialmente por una función que a cada
par (símbolo-leído, estado) le asocia: un
movimiento a izquierda o derecha en la
cinta; la escritura de un símbolo en el
nuevo cuadrado de la cinta; y un nuevo
estado."\textit{}

}}
      %\includegraphics[scale=1.0]{figurefile}
      \caption{}
      \label{}
   \end{figure}

\subsection{Conclusiones.}
Debido a este nuevo manejo de la información, se da la creación de los primeros ordenadores,que tenian la capacidad de realizar tareas como, combinar las diferentes simbologías,  contar, almacenar números enteros. Debido a esta evolución, nacen las necesidades de almacenar y optimizar los datos. 
El procesamiento de la información es el pilar de más importancia de nuestra sociedad, ya que el mundo se resume en tareas algoritmicas simples o muy complejas. Pero es importante conocer de donde nació y como evolucionó a través de diferentes momentos históricos y sociales la concepción del tratamiento de la información.



    






\addtolength{\textheight}{-12cm}   % This command serves to balance the column lengths
                                  % on the last page of the document manually. It shortens
                                  % the textheight of the last page by a suitable amount.
                                  % This command does not take effect until the next page
                                  % so it should come on the page before the last. Make
                                  % sure that you do not shorten the textheight too much.

%%%%%%%%%%%%%%%%%%%%%%%%%%%%%%%%%%%%%%%%%%%%%%%%%%%%%%%%%%%%%%%%%%%%%%%%%%%%%%%%



%%%%%%%%%%%%%%%%%%%%%%%%%%%%%%%%%%%%%%%%%%%%%%%%%%%%%%%%%%%%%%%%%%%%%%%%%%%%%%%%



%%%%%%%%%%%%%%%%%%%%%%%%%%%%%%%%%%%%%%%%%%%%%%%%%%%%%%%%%%%%%%%%%%%%%%%%%%%%%%%%





%%%%%%%%%%%%%%%%%%%%%%%%%%%%%%%%%%%%%%%%%%%%%%%%%%%%%%%%%%%%%%%%%%%%%%%%%%%%%%%%



\begin{thebibliography}{99}

\bibitem{c1} El infinito. S. Domínguez.

\bibitem{c2} 
Georg Cantor, el hombre que descubrió distintos infinitos. Marzo 2019.
\bibitem{c3} 
Crisis en los fundamentos de las matemática. A. Ortiz Fernández. 1988.

\bibitem{c4}Kurt Gödel y Alan Turing: una nueva mirada a los limites humanos. Claudio Gutiérrez.
\bibitem{c5} Kurt Gödel: La cumbre del imposible matemático. A. Martinón.
\bibitem{c6}  Diseño de un sistema criptográfico
a partir de la máquina enigma. J. Ortigosa. Enero 2007.
\bibitem{c7}  Alan Turing y los origenes de la investigacion multidisciplina. Agosto 2016.
\bibitem{c8}
Alan Turing y el nacimiento
de la inteligencia artificial. Antoni Escrig Vidal.
 Marzo 2007.

 \bibitem{c9} Informatica. Edgar Lopategui.  






\end{thebibliography}




\end{document}
